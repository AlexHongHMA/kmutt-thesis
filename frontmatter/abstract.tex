\newpage
\thispagestyle{plain}
\setcounter{page}{2}

\begin{center}
\begin{tabular}{ll}
Thesis Title & \parbox{10cm}{\thesistitle} \\
Thesis Credits & \thesiscredits \\
Candidate & \authorname \\
Thesis Advisor & \advisor \\
Program & \program \\
Field of Study & \fieldofstudy \\
Department & \department \\
Faculty & Engineering \\
Academic Year & \academicyear \\
\end{tabular}
\end{center}

\vspace{0.1cm}

% \section*{Abstract}
% \addcontentsline{toc}{chapter}{ABSTRACT}

\begin{center}
Abstract
\end{center}

\vspace{0.1cm}

Infrared imaging is vital for real-time methane leak detection in industrial environments, yet the computational complexity of 3D Convolutional Neural Networks (3D-CNNs) like VideoGasNet limits their deployment on resource-constrained systems. This paper proposes an optimised preprocessing pipeline for VideoGasNet, replacing the computationally intensive moving median background subtraction with a running average approach and incorporating image downscaling. Experiments on the GasVid dataset demonstrate that our method reduces preprocessing time significantly—by up to 66\% at reduced resolutions—while maintaining high classification accuracy (above 98\% even at quarter resolution). These enhancements enable real-time gas leak classification, offering a practical balance between accuracy and computational cost for industrial applications.

\noindent
\begin{tabular}{@{}p{2cm}p{12cm}@{}}
\textbf{Keywords:} & Infrared Imaging, Methane Emissions Detection, Running Average Background Subtraction, Deep Learning Optimisation, Computational Efficiency
\end{tabular}
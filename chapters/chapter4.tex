\chapter{RESULTS AND DISCUSSION}

To evaluate the performance and efficiency of our proposed solutions, we developed a comprehensive framework comparing [number] [method types]—[Method 1], [Method 2], and [Method 3]—at original [parameter] ([original value]) with [additional techniques] to reduce [error type] and improve [performance aspect]. After identifying [Method 2] as the most effective, we assessed its impact at different [parameters] and measured computational efficiency.

\section{Experimental Setup}

The experiments focused on [number] primary areas:
\begin{enumerate}
\item Comparing different [method types] ([Method 1], [Method 2], and [Method 3])
\item Evaluating the impact of [parameter] reduction ([value 1], [value 2], [value 3]) on model performance
\end{enumerate}

Performance was evaluated using [evaluation approach] for [class 1] versus [class 2] scenarios, with the [challenging scenario] being the most challenging and critical for practical applications. Each configuration was evaluated using metrics including [metric 1], [metric 2], [metric 3], and [metric 4], with particular attention to both [class 1] and [class 2] classification performance.

\section{[Method Types] Comparison}

\subsection{Visual Comparison}

% INSTRUCTIONS: Replace this with description of your comparison figures
% Example: Figure \ref{fig:method_comparison} illustrates the outcomes of various [processing techniques] using a [sample scenario]. The original [data type] shows [description]. When processed with different [method types], we observe that both [Method 1] and our [Method 2] better preserve the [important features] compared to [Method 3]. However, all three methods effectively [achieve goal].

% INSTRUCTIONS: Add your comparison figures here
% Replace this comment block with your own figure showing method comparisons
% Example structure:
% \begin{figure}[htbp]
% \centering
% \begin{subfigure}[t]{0.24\textwidth}
% \centering
% \includegraphics[width=\textwidth]{images/your_original_image.png}
% \caption{Original [data type]}
% \label{fig:original}
% \end{subfigure}%
% ... add more subfigures as needed
% \caption{Comparison of [method types] for [application]}
% \label{fig:method_comparison}
% \end{figure}

\subsection{Overall Classification Performance}

Table \ref{tab:classification_performance} shows the classification metrics for different [method types], emphasizing precision, recall, and F1-score for both [class 1] and [class 2] classes.

\begin{table}[htbp]
\caption{Classification Performance (\%) of Different [Method Types]}
\label{tab:classification_performance}
\begin{tabular}{|l|c|c|c|c|c|c|c|}
\hline
& \textbf{Overall} & \multicolumn{3}{c|}{\textbf{[Class 1]}} & \multicolumn{3}{c|}{\textbf{[Class 2]}} \\
\cline{3-8}
\textbf{Method} & \textbf{Accuracy} & \textbf{Prec.} & \textbf{Recall} & \textbf{F1} & \textbf{Prec.} & \textbf{Recall} & \textbf{F1} \\
\hline
[Method 2] & [Value 1] & [Value 2] & [Value 3] & [Value 4] & [Value 5] & [Value 6] & [Value 7] \\
\hline
[Method 1] & [Value 8] & [Value 9] & [Value 10] & [Value 11] & [Value 12] & [Value 13] & [Value 14] \\
\hline
[Method 3] & [Value 15] & [Value 16] & [Value 17] & [Value 18] & [Value 19] & [Value 20] & [Value 21] \\
\hline
\end{tabular}
\end{table}

The [Method 2] achieved the highest overall accuracy at [percentage]\%, outperforming both the [Method 1] ([percentage]\%) and [Method 3] ([percentage]\%). This improvement, while seemingly marginal, is considerable when combined with the benefits in computational speed.

\subsection{Confusion Matrix Analysis}

The confusion matrices in Table \ref{tab:confusion_matrix} provide insights into the classification performance of each method.

\begin{table}[htbp]
\caption{Confusion Matrix Analysis of [Method Types]}
\label{tab:confusion_matrix}
\begin{tabular}{|l|c|c|c|c|}
\hline
\textbf{Method} & \textbf{True} & \textbf{False} & \textbf{True} & \textbf{False} \\
& \textbf{[Class 1]} & \textbf{[Class 1]} & \textbf{[Class 2]} & \textbf{[Class 2]} \\
\hline
[Method 2] & [Value 1] & [Value 2] & [Value 3] & [Value 4] \\
\hline
[Method 1] & [Value 5] & [Value 6] & [Value 7] & [Value 8] \\
\hline
[Method 3] & [Value 9] & [Value 10] & [Value 11] & [Value 12] \\
\hline
\end{tabular}
\end{table}

The [Method 2] exhibits [performance description] with [error metric] [error type], a significant improvement over the [Method 3] ([error count] [error type]) and [Method 1] ([error count] [error type]). This [improvement description] is critical for field deployment, where [error type] can lead to [consequence]. Additionally, the [Method 2] achieves the lowest [error metric] ([count]) compared to [Method 3] ([count]) and [Method 1] ([count]), indicating superior [performance aspect].

\subsection{Pairwise Classification Performance}

To compare detection performance across different [scenarios], we conducted pairwise comparisons for [baseline] with each of the [test scenarios]. Table \ref{tab:pairwise_comparison} provides these results, with emphasis on the challenging [challenging pair] that represents [critical scenario].

\begin{table}[htbp]
\caption{Pairwise Accuracy Comparison (\%) Across Different [Method Types]}
\label{tab:pairwise_comparison}
\begin{tabular}{|l|c|c|c|c|c|c|c|}
\hline
\textbf{Method} & \textbf{[Pair 1]} & \textbf{[Pair 2]} & \textbf{[Pair 3]} & \textbf{[Pair 4]} & \textbf{[Pair 5]} & \textbf{[Pair 6]} & \textbf{[Pair 7]} \\
\hline
[Method 2] & [Value 1] & [Value 2] & [Value 3] & [Value 4] & [Value 5] & [Value 6] & [Value 7] \\
\hline
[Method 1] & [Value 8] & [Value 9] & [Value 10] & [Value 11] & [Value 12] & [Value 13] & [Value 14] \\
\hline
[Method 3] & [Value 15] & [Value 16] & [Value 17] & [Value 18] & [Value 19] & [Value 20] & [Value 21] \\
\hline
\end{tabular}
\end{table}

The [Method 2] demonstrates excellent performance on all pairwise comparisons, with [performance description] for most [scenario type] and [percentage]\% accuracy for the most challenging [challenging scenario], outperforming both [Method 3] ([percentage]\%) and [Method 1] ([percentage]\%) for this critical case.

\section{Computational Efficiency Analysis}

Table \ref{tab:processing_efficiency} compares the computational efficiency of the [method types] in terms of total [processing metric] and the average [time metric].

\begin{table}[htbp]
\caption{Computational Efficiency Comparison of [Method Types]}
\label{tab:processing_efficiency}
\begin{tabular}{|l|c|c|}
\hline
\textbf{Method} & \textbf{Average [Processing]} & \textbf{Average [Time Metric]} \\
& \textbf{Time ([Unit])} & \textbf{per [Unit] ([Time Unit])} \\
\hline
[Method 2] & [Value 1] & [Value 2] \\
[Method 1] & [Value 3] & [Value 4] \\
[Method 3] & [Value 5] & [Value 6] \\
\hline
\end{tabular}
\end{table}

The [Method 2] is computationally more efficient, taking on average [time value] [time unit] to process [data units], which is approximately [multiplier]× faster than the traditional [Method 1] with [parameter] ([time value] [time unit]) and [percentage]\% faster than the [Method 3] ([time value] [time unit]). This speedup in processing is invaluable for real-time applications, where rapid [processing] enables more immediate responses to [events].

The improved efficiency stems from the algorithmic simplicity of the [Method 2] approach—[process description] requires only the [current input] and [previous state], avoiding [expensive operation]. This significantly reduces both memory space and computational complexity without compromising high [performance metric].

\section{[Parameter] Scaling Results}

To assess the feasibility of deploying [application] systems on resource-constrained devices, we experimented with the impact of [parameter] reduction on model size and [performance metric].

\subsection{Model Size Comparison}

Table \ref{tab:model_size_comparison} compares key metrics across [number] [parameter] sizes using the [Method 2] for [processing].

\begin{table}[htbp]
\caption{Model Size Comparison Across Different [Parameter] Values}
\label{tab:model_size_comparison}
\begin{tabular}{|l|c|c|}
\hline
\textbf{[Parameter] Size} & \textbf{Parameters} & \textbf{Model Size} \\
\hline
[Value 1] & [Count 1] ([Size 1] MB) & [Size Value 1] KB \\
[Value 2] & [Count 2] ([Size 2] KB) & [Size Value 2] KB \\
[Value 3] & [Count 3] ([Size 3] KB) & [Size Value 3] KB \\
\hline
\end{tabular}
\end{table}

[Parameter] reduction results in dramatic decreases in model parameters and size—the [reduced version] model requires [percentage]\% fewer parameters than the [full version] model, while the [minimal version] model requires [percentage]\% fewer parameters. This reduction directly translates to lower memory requirements and computational costs, which are crucial factors for [deployment target].

\subsection{Classification Performance at Reduced [Parameters]}

Table \ref{tab:parameter_performance} presents the classification metrics for each [parameter value] using the [Method 2] technique.

\begin{table}[htbp]
\caption{Classification Performance (\%) Across Different [Parameter] Values}
\label{tab:parameter_performance}
\setlength{\tabcolsep}{5pt}
\centering
  \begin{tabular}{|l|c|c|c|c|c|c|c|}
  \hline
  \multirow{2}{*}{\textbf{Parameter}} & \textbf{Overall} & \multicolumn{3}{c|}{\textbf{Class 1}} & \multicolumn{3}{c|}{\textbf{Class 2}} \\
  \cline{3-8}
   & \textbf{Accuracy} & \textbf{Prec.} & \textbf{Recall} & \textbf{F1} & \textbf{Prec.} & \textbf{Recall} & \textbf{F1} \\
  \hline
  Value 1 & XX.X & XX.X & XX.X & XX.X & XX.X & XX.X & XX.X \\
  \hline
  Value 2 & XX.X & XX.X & XX.X & XX.X & XX.X & XX.X & XX.X \\
  \hline
  Value 3 & XX.X & XX.X & XX.X & XX.X & XX.X & XX.X & XX.X \\
  \hline
  \end{tabular}
\end{table}

The results show a gradual performance decline as the [parameter] [changes], with overall accuracy [changing] from [percentage]\% at [baseline] to [percentage]\% at [reduced level] and [percentage]\% at [minimal level]. Most significantly, [performance aspect] is impacted most with [parameter change], from [percentage]\% at [baseline] to [percentage]\% at [minimal level].

Nevertheless, even at [minimal level], the model achieves over [percentage]\% accuracy in classifying all samples, displaying remarkable robustness to [parameter] reduction.

\subsection{Pairwise Comparison Across [Parameter] Values}

Table \ref{tab:pairwise_parameter} presents the pairwise accuracy comparison for different [parameter] sizes, highlighting the impact of [parameter] on detecting [scenarios] of varying [characteristics].

\begin{table}[htbp]
\caption{Pairwise Accuracy Comparison (\%) Across Different [Parameter] Values}
\label{tab:pairwise_parameter}
\begin{tabular}{|l|c|c|c|c|c|c|c|}
\hline
\textbf{[Parameter]} & \textbf{[Pair 1]} & \textbf{[Pair 2]} & \textbf{[Pair 3]} & \textbf{[Pair 4]} & \textbf{[Pair 5]} & \textbf{[Pair 6]} & \textbf{[Pair 7]} \\
\hline
[Value 1] & [Perf 1] & [Perf 2] & [Perf 3] & [Perf 4] & [Perf 5] & [Perf 6] & [Perf 7] \\
\hline
[Value 2] & [Perf 8] & [Perf 9] & [Perf 10] & [Perf 11] & [Perf 12] & [Perf 13] & [Perf 14] \\
\hline
[Value 3] & [Perf 15] & [Perf 16] & [Perf 17] & [Perf 18] & [Perf 19] & [Perf 20] & [Perf 21] \\
\hline
\end{tabular}
\end{table}

The pairwise comparison reveals an important aspect: [parameter] reduction primarily impacts the [detection type] ([challenging pair]), where accuracy [changes] from [percentage]\% at [baseline] to [percentage]\% at [reduced level] and [percentage]\% at [minimal level]. Detection of [other scenarios] is remarkably robust, with accuracy consistently above [percentage]\% even at [minimal level].

This pattern suggests that [challenging scenarios] have more [characteristics] that require [high parameter value] for reliable [detection/processing], while [easier scenarios] produce more [prominent features] that remain [detectable/processable] even at [low parameter values].

\section{Error Analysis}

To provide deeper insights into system limitations and guide practical deployment decisions, we conducted comprehensive error analysis across [parameter values] using the [Method 2] to understand system limitations and guide deployment decisions. 
% INSTRUCTIONS: Add reference to your error analysis figures here
% Example: Figure \ref{fig:error_analysis} shows representative error cases at different [parameter levels].

% INSTRUCTIONS: Add your error analysis figures here
% Replace this comment block with your own figures showing error cases
% Example structure:
% \begin{figure}[htbp]
% \centering
% \begin{subfigure}[t]{0.31\textwidth}
% \centering
% \includegraphics[width=\textwidth]{images/your_error_case1.png}
% \caption{[Error Type 1]: [Description]}
% \label{fig:error1}
% \end{subfigure}
% ... add more subfigures as needed
% \caption{Representative error cases showing [challenges]}
% \label{fig:error_analysis}
% \end{figure}

\subsection{[Parameter Level 1] Error Analysis}

At [parameter level 1], the system achieved [error metric] with [error count] [error type] ([percentage]\%), primarily from [scenario type] with [characteristics]. These [missed cases] involve [description], representing challenging cases for automated [processing]. The [error type] cases typically occur when [conditions].

\subsection{[Parameter Level 2] Error Analysis}

The [parameter level 2] model introduced [error count] [error type] ([percentage]\%) from [error cause] and [error count] [error type] ([percentage]\%), mainly affecting [affected area]. [Parameter] reduction impacts the system's ability to [capability]. The increase in [error type] is primarily attributed to [cause].

\subsection{[Parameter Level 3] Error Analysis}

Error rates increased significantly to [error count] [error type] ([percentage]\%) and [error count] [error type] ([percentage]\%) at [parameter level 3]. [Error type] demonstrate [characteristic] when [condition], while [error type] confirm [degraded capability]. The system begins to [behavior] due to [cause].

\subsection{Error Pattern Analysis}

The analysis reveals a clear trade-off between [efficiency] and [performance]:

\begin{enumerate}
\item \textbf{[Error Type 1]:} Primarily occur with [characteristics] that present [issue] for reliable [processing]. These cases represent the fundamental limitation of [approach] when dealing with [extreme cases].

\item \textbf{[Error Type 2]:} Occur when [background features] are misinterpreted as [target], particularly at [parameter levels] where [detail] is insufficient for proper [discrimination].

\item \textbf{[Parameter] Impact:} [High parameter] ensures maximum [performance aspect] for [critical applications], while [medium parameter] provides optimal balance for [use case] with reduced [cost].
\end{enumerate}

\section{Discussion}

\subsection{[Method Types] Performance}

The comparative analysis across different [method types] reveals several key insights:

\begin{enumerate}
\item \textbf{Superior [Performance Aspect]:} The [Method 2] achieved the highest overall [metric] ([percentage]\%) with exceptional [performance measure] and the lowest [error metric].

\item \textbf{Computational Efficiency:} The [Method 2] demonstrated the best computational efficiency with [improvement factor]× faster processing compared to [Method 1], making it suitable for [applications].

\item \textbf{Adaptive Capability:} The [characteristic] of [Method 2] effectively accounts for [environmental changes].
\end{enumerate}

\subsection{[Parameter] Trade-offs}

The study revealed important insights regarding the relationship between [parameter] and [performance]:

\begin{enumerate}
\item \textbf{Optimal Balance:} [Medium parameter] offer a favorable balance between [efficiency] and [performance], with only [percentage]\% [metric] decrease.

\item \textbf{[Scenario] Dependency:} The impact of [parameter] reduction is highly dependent on [scenario type], with [challenging scenarios] being more affected than [easier scenarios].

\item \textbf{Practical Deployment:} [Minimal parameter] models maintain excellent performance for [specific scenarios] while maximizing [efficiency].
\end{enumerate}

\subsection{Practical Implications}

The findings have significant implications for real-world deployment:

\begin{enumerate}
\item \textbf{Real-time Capability:} The [improvement factor]× improvement in [processing metric] enables [real-time capability] in [environments].

\item \textbf{Cost-effective Implementation:} [Performance achievement] reduce [costs] and [operational impacts].

\item \textbf{Edge Device Deployment:} Model size reduction of up to [percentage]\% enables deployment on [target devices].

\item \textbf{Scalable Solutions:} The optimized approach can be scaled across [multiple deployment scenarios].
\end{enumerate}